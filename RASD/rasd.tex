\documentclass[10pt]{article}
\usepackage[utf8]{inputenc}
\usepackage{listings}
\usepackage{xcolor}
\usepackage{subcaption}

\definecolor{codegreen}{rgb}{0,0.6,0}
\definecolor{codegray}{rgb}{0.5,0.5,0.5}
\definecolor{codepurple}{rgb}{0.58,0,0.82}
\definecolor{backcolour}{rgb}{0.95,0.95,0.92}
\usepackage{color}   %May be necessary if you want to color links
\usepackage{hyperref}
\usepackage{graphicx}
\graphicspath{ {./images/} }
\hypersetup{
    colorlinks=true, %set true if you want colored links
    linktoc=all,     %set to all if you want both sections and subsections linked
    linkcolor=black,  %choose some color if you want links to stand out
    urlcolor=blue,
}
\lstdefinestyle{mystyle}{
    backgroundcolor=\color{backcolour},   
    commentstyle=\color{codegreen},
    keywordstyle=\color{magenta},
    numberstyle=\tiny\color{codegray},
    stringstyle=\color{codepurple},
    basicstyle=\ttfamily\footnotesize,
    breakatwhitespace=false,         
    breaklines=true,                 
    captionpos=b,                    
    keepspaces=true,                 
    numbers=left,                    
    numbersep=5pt,                  
    showspaces=false,                
    showstringspaces=false,
    showtabs=false,                  
    tabsize=2
}

\lstset{style=mystyle}


\title{RASD}
\author{Mauro Famà, Giacomo Lombardo}
%\date{October 2021}

\begin{document}
\thispagestyle{empty}
\begin{titlepage}
    \begin{center}
       %\vspace*{2cm}
       {\Huge \textbf{RASD}} %%Replace this with the Title of your research
       \vspace{0.5cm}
       \\
    \begin{LARGE}
        {Requirements Analysis and Specification Document}
        \vspace{1.0cm}
        \\
        \includegraphics[width=13cm]{polimi.png}
        \vspace{1.5cm}\\
        Mauro Famà (10631287)\\Giacomo Lombardo (10674987)
        \vspace{1.5cm}\\
        {A.Y. 2021/2022}
    \end{LARGE}  
   \end{center}
\end{titlepage}
\newpage
\tableofcontents %this command creates an index
\newpage
\section{Introduction}
As world's population is increasing at steady pace, new challenges arise with its growth.
Accordingly to a recent UN estimate, by 2050 globally there will be almost 10 billion people, 
and food demand is expected to increase between 59\% to 98\%. Furthermore, climate change is causing 
problems to the agriculture that are getting bigger every year: its effect is predicted to result in a 
4\%-26\% loss by the end of the century in the Indian agricuture sector.\\
To tackle the situation, Telangana's state, which is the 11th biggest state in India, has decided to develop a 
platform to implement anticipatory, data-driven models to strengthen the policies in farming, with the ultimate
goal of increasing the output of the agriculture sector.\\
This is the idea behind DREAM (Data-dRiven prEdictive fArMing), a platform where Telangana's
Policy Makers, farmers and agronomists cooperate in order to enhance agriculture performances and facilitate the 
management of help requests, visits to farms and general communications betweens the actors in this scenario.
\subsection{Purpose}
The purpose of this document is to present a detailed description of the DREAM
 (Data-dRiven PrEdictive FArMing) platform. It will explain the purpose and features of 
 the software, the interfaces of the software, what the software will do and the constraints
  under which it must operate. This document is intended for software users and also 
  potential developers. The description will be offered by using models, providing scenarios and 
  their relative use cases, analyzing most important functional and non-functional requirements.
\subsection{Scope} %qui inserirei overview of the function of the system and the reasons for its development, goals and success criteria
The DREAM platform is intended for use by Telengana's policy makers, farmers and agronomists.\\
DREAM allows Telangana's policy makers, through the use of data provided by farmers regarding goods 
production, to extrapolate significative information and obtain a general view of the food production of the entire region. 
In this way, they will be given a platform to model and monitor the region's food system, facilitating the study and the 
realization of an adequate and well-performing long-term production strategy. The platform will provide policy makers the possibility 
to easily monitor farmers' performance and manage agronomists' interventions whenever and wherever needed. 
Furthermore, they will be able to identify those farmers whose production has been particularly good or bad,
in order to reward the best performing farmers with incentives and to provide help to the farmers in need.\\ 
DREAM's goal is also to provide useful resources and tools to local farmers, who will be given a platform to facilitate 
the farmer-to-farmer and farmer-to-policy maker communication. Via DREAM, farmers will be able to participate to forum
discussions, easily create help requests and obtain sensitive information such as accurate weather forecasts and personalized 
suggestions regarding goods and agriculture routines.\\
For what concerns the local agronomists, DREAM facilitates the daily visit controls scheduling and regulates it accordingly
to farmer needs, increasing the visits whether a farmer performance is significantly lower than the average.\\
In this document, the agronomist perspective won't be analyzed in detail, but it will be considered only marginally and in those
cases where there is a significant interaction between agronomists and the other actors.
\subsubsection{Goals}
    \begin{center}
        \vspace{0.5cm}
        \begin{tabular}{|c c|} 
        \hline
        Goal & Description \\ 
        \hline\hline
        G1 & Allow farmers to receive assistance \\ 
        \hline
        G2 & Allow Telangana's Policy Makers to monitor local agriculture\\ 
        \hline
        G3 & Send incentives and various benefits to the best local farmers\\ 
        \hline
        G4 & Create a farmer's network to promote collaboration and mutual aid \\ 
        \hline
        G5 & Facilitate the management of agronomists's schedules \\ 
        \hline
        \end{tabular}
    \end{center}
\subsubsection{World and Machine Phenomena}
\begin{center}
    \vspace{0.5cm}
    \begin{tabular}{|c c|} 
    \hline
    World Phenomena & Description \\ 
    \hline\hline
    WP1 & Crop damage due to natural causes \\ 
    \hline
    WP2 & Heavy meteorological conditions\\ 
    \hline
    WP3 & Agriculture operations and routines\\ 
    \hline
    WP4 & Agricuture product collection \\ 
    \hline
    WP5 & Control visits to farmers \\ 
    \hline
    \end{tabular}
    \vspace{0.5cm}\\
    \begin{tabular}{|c c c|} 
        \hline
        Shared Phenomena & Description & Controlled By\\ 
        \hline\hline
        SP1 & Insertion of crop collection data & W \\ 
        \hline
        SP2 & Visualization of weather forecasts & M \\ 
        \hline
        SP3 & Visualization of suggestions & M \\ 
        \hline
        SP4 & Sending of notifications by TPMs & W \\ 
        \hline
        SP5 & Receiving of notifications by farmers & M \\ 
        \hline
        SP6 & Sending help requests & W \\ 
        \hline
        SP7 & Contribution to a forum discussion & W \\ 
        \hline
        SP8 & Visualization of crop collection data & M \\ 
        \hline
        SP9 & Interview request & W \\ 
        \hline
        SP10 & Agronomist visit request & W \\ 
        \hline
        SP11 & Login/registration on the platform & W \\ 
        \hline
        \end{tabular}
\end{center}
\subsection{Definitions, Acronyms, Abbreviations}
\subsection{Revision history}
\subsection{Reference Documents}
\subsection{Document Structure}
\newpage
\section{Overall Description}
\subsection{Product Perspective}
\subsubsection{Scenarios}
\begin{itemize}
    \item \textit{Insertion of crop collection data into the system.}\\ 
    Rajesh is a Telangana's local farmer. Every morning he wakes up, does a little and poor breakfast and goes to work in the fields.
    Although being a very humble and physically demanding work, agriculture has always been a priority in Rajesh life: nothing gives him
    more satisfaction than seeing the results of his hard work and when he goes to sleep at night, even though he is tired from the long day of work,
    he falls asleep with a smile on his mouth thinking about his crop happily growing in the night. When the crop is ready to be collected, the atmosphere
    is electric in Rajesh's farm: if the crop is good and particularly abundant, all the workers have a big dinner, finely prepared by Rajesh and his wife,
    followed by a long party night where everybody sings and dances. The day after the celebration, Rajesh wakes up with a tremendous hangover, and proceeds
    to login into the DREAM platform. Once he has logged in, he proceeeds to insert into the system the data regarding the collected crop.
    He would have preferred to sleep a few more hours, and only by looking at the computer's screen he gets a big headache, but he carries on, because he
    remembers that tomorrow will be another happy day in the fields of Telangana.
\end{itemize}
\end{document}
