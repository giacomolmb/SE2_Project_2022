\documentclass[10pt]{article}
\usepackage[utf8]{inputenc}
\usepackage{listings}
\usepackage{xcolor}
\usepackage{subcaption}

\definecolor{codegreen}{rgb}{0,0.6,0}
\definecolor{codegray}{rgb}{0.5,0.5,0.5}
\definecolor{codepurple}{rgb}{0.58,0,0.82}
\definecolor{backcolour}{rgb}{0.95,0.95,0.92}
\usepackage{color}   %May be necessary if you want to color links
\usepackage{hyperref}
\usepackage{graphicx}
\graphicspath{ {./images/} }
\hypersetup{
    colorlinks=true, %set true if you want colored links
    linktoc=all,     %set to all if you want both sections and subsections linked
    linkcolor=black,  %choose some color if you want links to stand out
    urlcolor=blue,
}
\lstdefinestyle{mystyle}{
    backgroundcolor=\color{backcolour},   
    commentstyle=\color{codegreen},
    keywordstyle=\color{magenta},
    numberstyle=\tiny\color{codegray},
    stringstyle=\color{codepurple},
    basicstyle=\ttfamily\footnotesize,
    breakatwhitespace=false,         
    breaklines=true,                 
    captionpos=b,                    
    keepspaces=true,                 
    numbers=left,                    
    numbersep=5pt,                  
    showspaces=false,                
    showstringspaces=false,
    showtabs=false,                  
    tabsize=2
}

\lstset{style=mystyle}


\title{RASD}
\author{Mauro Famà, Giacomo Lombardo}
%\date{October 2021}

\begin{document}
\thispagestyle{empty}
\begin{titlepage}
    \begin{center}
       %\vspace*{2cm}
       {\Huge \textbf{RASD}} %%Replace this with the Title of your research
       \vspace{0.5cm}
       \\
    \begin{LARGE}
        {Requirements Analysis and Specification Document}
        \vspace{1.0cm}
        \\
        \includegraphics[width=13cm]{polimi.png}
        \vspace{1.5cm}\\
        Mauro Famà (10631287)\\Giacomo Lombardo (10674987)
        \vspace{1.5cm}\\
        {A.Y. 2021/2022}
    \end{LARGE}  
   \end{center}
\end{titlepage}
\newpage
\tableofcontents %this command creates an index
\newpage
\section{Introduction}
As world's population is increasing at steady pace, new challenges arise with its growth.
Accordingly to a recent UN estimate, by 2050 globally there will be almost 10 billion people, 
and food demand is expected to increase between 59\% to 98\%. Furthermore, climate change is causing 
problems to the agriculture that are getting bigger every year: its effect is predicted to result in a 
4\%-26\% loss by the end of the century in the Indian agricuture sector.\\
To tackle the situation, Telangana's state, which is the 11th biggest state in India, has decided to develop a 
platform to implement anticipatory, data-driven models to strengthen the policies in farming, with the ultimate
goal of increasing the output of the agriculture sector.
\subsection{Purpose}
The purpose of this document is to present a detailed description of the DREAM
 (Data-dRiven PrEdictive FArMing) platform. It will explain the purpose and features of 
 the software, the interfaces of the software, what the software will do and the constraints
  under which it must operate. This document is intended for software users and also 
  potential developers. The description will be offered by using models, providing scenarios and 
  their relative use cases, analyzing most important functional and non-functional requirements.
\subsubsection{Goals}
    \begin{center}
        \begin{tabular}{|c c|} 
        \hline
        Goal & Description \\ 
        \hline\hline
        G1 & Allow farmers to receive assistance \\ 
        \hline
        G2 & Allow Telangana's Policy Makers to monitor local agriculture\\ 
        \hline
        G3 & Send incentives and various benefits to the best local farmers\\ 
        \hline
        G4 & Create a farmer's network to promote collaboration and mutual aid \\ 
        \hline
        G5 & Facilitate the management of agronomists's schedules \\ 
        \hline
        \end{tabular}
    \end{center}
\subsection{Scope} %qui inserirei overview of the function of the system and the reasons for its development, goals and success criteria
The DREAM platform is intended for use by Telengana's policy makers and farmers. 
The reason that policy makers need this platform is that through the use of information 
provided by farmers on the type and quantity of goods produced, aggregated, and combined with 
information about land use, such as the amount of water used for irrigation or soil moisture, 
it will be possible to extrapolate data capable of modeling the food system of the entire region, 
giving the possibility to consult aggregate data to facilitate the study of an adequate, 
performing and long-term production strategy.\\
Thanks to this platform, farmers will be able to exploit Telengana's technological infrastructure 
to improve their production activity. They will receive personalized advice and news about 
their area of work and will be able to request personalized assistance from agronomists and 
compare themselves with other farmers in the area. To receive these benefits, farmers will be 
required to consistently enter information about their products and the problems they have to face.
\subsubsection{World and Machine Phenomena}
\begin{center}
    \begin{tabular}{|c c|} 
    \hline
    World Phenomena & Description \\ 
    \hline\hline
    WP1 & Crop damage due to natural causes \\ 
    \hline
    WP2 & Heavy meteorological conditions\\ 
    \hline
    WP3 & Agriculture operations and routines\\ 
    \hline
    WP4 & Agricuture product collection \\ 
    \hline
    WP5 & Control visits to farmers \\ 
    \hline
    \end{tabular}
\end{center}
\subsection{Definitions, Acronyms, Abbreviations}
\subsection{Revision history}
\subsection{Reference Documents}
\subsection{Document Structure}
\newpage
\section{Hypothesis}
The database is implemented under the following hypothesis:
\begin{itemize}
    \item Hypothesis 1
    \item Hypothesis 2
    \item Et cetera
\end{itemize}
\end{document}
